\documentclass{article}
\usepackage[utf8]{inputenc}
\usepackage[margin=2cm]{geometry}
\usepackage{amsmath, amsfonts, amssymb, amsthm}
\usepackage{hyperref}

\newtheorem{thm}{Theorem}[section]
\newtheorem{lem}[thm]{Lemma}
\newtheorem{prop}[thm]{Proposition}
\newtheorem{cor}[thm]{Corollary}
\newtheorem{conj}[thm]{Conjecture}
\newtheorem{exmp}[thm]{Example}

\theoremstyle{definition}
\newtheorem{defn}{Definition}[section]


\title{Partial summary of work done in summer of 2016}
\author{Sayantan Khan}

\newcommand{\integer}{\mathbb{Z}}
\newcommand{\znz}{\mathbb{Z}/N\mathbb{Z}}
\newcommand{\indi}{\mathbf{1}_A}
\newcommand{\indim}{\mathbf{1}_{M_A}}
\newcommand{\vep}{\varepsilon}

\begin{document}

\maketitle

\section{Heating a disc}
Laplace equation. Solution to Laplace equation with the appropriate boundary conditions leads naturally to Fourier series. Questions of convergence raised. Must be answered in pieces.

\section{The fourier series}
Rather than dealing with regular convergence, dealt with Abel summability of the fourier series. Poisson kernel.

\section{Digression: All about kernels}
Convolution operation. Kernels. Dirac sequences of kernels. Proved that $f \ast D_n$ converges uniformly to $f$.

\section{Weaker notions of convergence}
Master theorem showed that the fourier series is Abel summable. Now create similar kernels for finite fourier series (Dirichlet kernel) and averages of first $n$ terms (Fejér kernel). Fejér kernels form a Dirac sequence, hence the fourier series is Cesàro summable.

\section{Orthonormal basis for $C(T)$}
Proved exponential polynomials dense in $C(T)$: two different proofs using Poisson and Fejér kernels (Fejér kernel gives explicit approximation). Two line proof using Stone-Weirstrass, and a much nastier proof using Weirstrass approximation.

\section{Strengthening the conditions on $f$}
Used density result to show $\displaystyle \lim_{n \rightarrow \infty} \hat{f}(n) = 0$ for a continuous $f$ (Riemann-Lebesgue lemma). Proved the principle of localisation. Then showed that if $\hat{f}(n)$ is $O\left( \frac{1}{n} \right)$, then the fourier series converges. Subsequently showed that if $f \in C^1(T)$, then $\hat{f}(n)$ is $o\left( \frac{1}{n} \right)$

\section{Computing the zeta function for positive even integers}
Used the result that $x^2 \in C^1(T)$, hence it's fourier series converges at $x=0$ to get
\begin{align*}
    \sum_{n=1}^{\infty} (-1)^{n+1}\frac{1}{n^2} = \frac{\pi^2}{8}
\end{align*}
And a consequence of this result is that $\zeta(2) = \frac{\pi^2}{6}$. Similarly, by computing the fourier coefficients of $x^{2k}$, one can compute $\zeta(2k)$.

\section{Alternative formula for $\zeta(s)$ where $s > 1$}
An alternative formula for $\zeta(s)$ when $s>1$ is given by
\begin{align*}
    \zeta(s) = \prod_{p \in \mathcal{P}} \frac{1}{1 - p^{-s}}
\end{align*}
where $\mathcal{P}$ is the set of prime numbers.

One can prove this using the fundamental theorem of arithmetic.

\section{Proving Weirstrass approximation theorem from Fejér's theorem}
From Fejér's theorem we got that trigonometric polynomials are dense in $C(T)$. It will then suffice to show that $\cos(n\theta)$ and $\sin(n\theta)$ can be approximated using polynomials on $T$. Consider the $m$\textsuperscript{th} Taylor polynomial for these functions and look at the remainder. For sufficiently large $m$, the remainder can be made smaller than a given $\varepsilon > 0$. This gives a polynomial approximation for the function and proves the proposition.

\section{Showing $\bar{z}$ cannot be uniformly approximated by polynomials in $z$ in a compact set in $\mathbb{C}$}
Without loss of generality, assume the compact set in question contains the unit circle (if it doesn't, rescaling and translation should do the trick). Now assume some polynomial $p$ $\varepsilon$ approximates $\bar{z}$ where $\varepsilon < 0.5$. In that case
\begin{align*}
    |zp(z) - z\bar{z}| < |z|\varepsilon
\end{align*}
In particular, on the unit circle, the inequality reduces to
\begin{align*}
    |zp(z) - 1| < \varepsilon
\end{align*}
This would mean for all points $x$ on the circle, the real part of $xp(x)$ lies between $0.5$ and $1.5$. But notice that if take $2(n!)$ equally spaced points on the circle, where $n$ is the degree of $zp(z)$, then the sum of $zp(z)$ over those points is $0$, which means the real part of $zp(z)$ on at least one of those points must be less than or equal to $0$. We have a contradiction.

\section{Poisson summation of the normal distribution}
The task was to show
\begin{align*}
    \frac{1}{\sqrt{2\pi}} \int_{\mathbb{R}} \exp\left(-\frac{x^2}{2} + i\lambda x\right) dx = \exp\left( -\frac{\lambda^2}{2} \right)
\end{align*}

Denote the value of the integral by $I(\lambda)$. The integral can be evaluated by first noting the imaginary part of the function inside the integral is odd; it goes to $0$. Performing integration by parts on the real part of the function, one notices that $I(\lambda)$ satisfies the following differential equation:
\begin{align*}
    \frac{dI}{d\lambda} = -\lambda I
\end{align*}
This gives the required expression.

\section{Hausdorff moment theorem}
The Hausdorff moment theorem states that if for continuous function $f$ and $g$ and a compact interval $I$, the following equation holds:
\begin{align*}
    \int_{I}x^nf(x) dx = \int_{I}x^ng(x) dx
\end{align*}
for all non-negative integers $n$, then $f \equiv g$ on $I$.

This problem is equivalent to showing $k \equiv 0$ where $k =g-f$ which can be done by showing
\begin{align*}
    \int_{I}k^2(x) dx = 0
\end{align*}
Let $p$ be an $\varepsilon$ polynomial approximation of $k$. Then
\begin{align*}
    \left|\int_{I}k^2(x) dx - \int_{I}p(x) k(x) dx \right| < |I|\varepsilon
\end{align*}
But by the hypothesis, $\displaystyle \int_{I}p(x)k(x) dx = 0$ for all polynomials $p$. This completes the proof.

\section{Filler}
Fill missing stuff here

\section{Roth's Theorem}
In very loose terms, Roth's theorem states that for a given number $0 <\delta \leq 1$, also called density, there exists a natural number $N$ such that any subset $A$ of $[N]$\footnote{$[N]$ is shorthand for the set $\{0, 1, \ldots, N-1\}$.} which has cardinality more than or equal to $\delta N$ contains a three term AP.

Clearly, if the density is $1$, then the statement is quite obvious. What the following proof does is given a set $[N]$ and its subset $A$, it tries to find a large enough subset $N'$ of $[N]$ such that $N'$ is also an AP, and the density of $A \cap N'$ in $N'$ is more than $\delta$ by some fixed multiplicative factor. Once we have this capability, we iterate until the density of $A$ in some subprogression exceeds $1$, in which case we're done.

What we do first is identify the set $[N]$ with the group $\mathbb{Z}/N\mathbb{Z}$ in the natural way. This way, we can also identify the subset $A$ as the subset of the group. Consider the function $\mathbf{1}_A$ from $\mathbb{Z}/N\mathbb{Z}$ to $\mathbb{C}$ which is $1$ is the argument is in $A$, otherwise $0$. Now we can do some fourier analysis since we have an $L_2$ function.

Consider all the elements $x,y,z$ of $A$ such that $x+z = 2y$. Clearly, these form a 3-AP in $\mathbb{Z}/N\mathbb{Z}$. These $\mathbb{Z}/N\mathbb{Z}$ progressions need not be $\integer$ progressions. But let's find a way of counting these $\znz$ progressions nevertheless.

Consider the following function:
\begin{align*}
    f(x,y,z) = \frac{1}{N}\sum_{k=0}^{N-1} \exp\left(-\frac{2\pi i}{N}(x -2y +z)k\right)
\end{align*}
Clearly, this function will be $1$ if $(x,y,z)$ form a $\znz$ progression, otherwise it will be $0$. Hence the number of such progressions is given by:
\begin{align*}
    S_0 &= \sum_{x,y,z \in A}f(x,y,z) \\
    &= \sum_{x,y,z \in A} \frac{1}{N}\sum_{k=0}^{N-1} \exp\left(-\frac{2\pi i}{N}(x -2y +z)k\right) \\
    &= \sum_{x,y,z \in [N]} \indi(x)\indi(y)\indi(z) \frac{1}{N}\sum_{k=0}^{N-1} \exp\left(-\frac{2\pi i}{N}(x -2y +z)k\right)
\end{align*}
Since the sums are finite, with appropriate rearrangement, we get:
\begin{align*}
    S_0 = \frac{1}{N} \sum_{k=0}^{N-1} \widehat{\indi}(k)^2 \widehat{\indi}(-2k)
\end{align*}
where $\widehat{f}(k)$ is the $k$\textsuperscript{th} fourier coefficient of $f$, which is defined by:
\begin{align*}
    \widehat{f}(k) = \sum_{x=0}^{N-1} f(x) \exp\left(-\frac{2\pi i}{N}kx\right)    
\end{align*}
It is then obvious that $\widehat{\indi}(0) = \delta N$. The original sum then becomes:
\begin{align*}
    S_0 = \delta^3 N^2 + \frac{1}{N} \sum_{k=1}^{N-1} \widehat{\indi}(k)^2 \widehat{\indi}(-2k)
\end{align*}
Without loss of generality, assume that $N$ is odd so that $-2k$ is never $0$, as $k$ goes from $1$ to $N-1$. If the $\widehat{\indi}(k)$ are bounded by a small enough number, then perhaps we can say something about the number of $\znz$ progressions in $A$. Notice that the sum $S_0$ also counts the trivial progressions. Hence the number of non-trivial progressions is $S_0 - \delta N$.

Clearly, the bound on $\widehat{\indi}(k)$ depends upon $\indi$, which in turn depends upon the set $A$. We call a set $\vep$-uniform if $\widehat{\indi}(k) \leq \vep N$ for all non-zero $k$. Also, define $M_A$ to be the set $\left[\frac{N}{3}, \frac{2N}{3}\right) \cap A$. Clearly, if $x$ and $y$ belong to $M_A$, and $x,y,z$ form a $\znz$ progression, then they also form a $\integer$ progression.

Now assume $A$ is $\vep$-uniform for $\vep < \frac{\delta^2}{8}$ and if $|M_A| \geq \frac{\delta N}{4}$, then the number of $\integer$ progressions $S$ is atleast $\frac{\delta^3 N^2}{32}$. Doing the same thing as we did for computing $S_0$, we get:
\begin{align*}
    S &\geq \frac{1}{N} \sum_{k=0}^{N-1} \widehat{\indim}(k)\widehat{\indi}(k)\widehat{\indim}(-2k) \\
    &= \delta |M_A|^2 + \frac{1}{N} \sum_{k=1}^{N-1} \widehat{\indim}(k)\widehat{\indi}(k)\widehat{\indim}(-2k)
\end{align*}
Let's try to bound the second term in the sum above.
\begin{align*}
    \left| \sum_{k=1}^{N-1} \widehat{\indim}(k)\widehat{\indi}(k)\widehat{\indim}(-2k) \right| &\leq \vep N \sum_{k=1}^{N-1} \left|\widehat{\indim}(k)\widehat{\indim}(-2k)\right| \\
    &\leq \vep N \left( \sum_{k=1}^{N-1} \left| \widehat{\indim}(k) \right|^2 \right)^{\frac{1}{2}} \left( \sum_{k=1}^{N-1} \left| \widehat{\indim}(-2k) \right|^2 \right)^{\frac{1}{2}} &&\text{(Cauchy-Schwarz inequality)} \\
    &\leq \vep N \sum_{k=0}^{N-1} \left| \widehat{\indim}(k) \right|^2 \\
    &= \vep N^2 \sum_{k=0}^{N-1} |\indim(k)|^2 &&\text{(Plancherel's equality)} \\
    &= \vep N^2 |M_A|
\end{align*}
Now we use the inequalities we took for our hypotheses, and we get:
\begin{align*}
    S \geq \delta|M_A|^2 - \vep N|M_A| \geq \frac{\delta^3 N^2}{32}
\end{align*}
However, note that $S$ also counts the trivial progressions. There are exactly $\delta N$ trivial progressions. Hence if $\frac{\delta^3 N^2}{32} - \delta N > 0$, only then will the set contain a 3-AP. That means if $N > \frac{32}{\delta^2}$, then the set contains a three term AP.

We have shown that if $N > \frac{32}{\delta^2}$, $A$ is $\vep$-uniform for $\vep < \frac{\delta^2}{8}$, and $|M_A| \geq \frac{\delta N}{4}$, then $A$ contains a three term $\integer$ progression.

On the contrary, assume that $A$ does not contain a 3-AP. Then one of the above conditions must be unsatisfied. We can ignore the first condition, since we can make $N$ as large as we want, given a fixed $\delta$. That leaves the other two conditions. We'll show if either of those conditions are unsatisfied, then $[N]$ contains a sub-progression $P$, such that the $|P| \geq \frac{\delta^2 \sqrt{N}}{256}$ such that the density of $A \cap P$ is more than $\delta + \frac{\delta^2}{64}$.

If $|M_A| < \frac{\delta N}{4}$, then consider the sets $L_A$ and $R_A$ defined as follows:
\begin{align*}
    L_A &= A \cap \left[ 0, \frac{N}{3} \right) \\
    R_A &= A \cap \left[ \frac{2N}{3}, N \right)
\end{align*}
The density of $A$ in one of these sets will be greater than or equal to $\frac{9\delta}{8}$, and these sets are subprogressions of $[N]$ of length $\frac{N}{3}$. This proves the theorem if $|M_A| < \frac{\delta N}{4}$.

Now consider what happens if the second condition is not satisfied, i.e. if $|\widehat{\indi}(r)| \geq \vep N$, where $\vep = \frac{\delta^2}{8}$ for some $r$. We'll need the following lemma to progress:
\begin{lem}
    If $|\widehat{\indi}(r)| \geq \vep N$, then there exists a non-overlapping $\znz$ progression $B$ such that $|B| > \frac{\sqrt{N}}{4}$ such that the density of $A \cap B$ in $B$ is $\delta + \frac{\vep}{4}$. 
\end{lem}

\begin{proof}
    It's rather long, and I don't feel like writing it. It uses a bit of fourier analysis though. I'll write it formally soon.
\end{proof}

That does it, doesn't it?
\end{document}
